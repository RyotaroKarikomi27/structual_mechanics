\section{はじめに}
  \subsection{片持ち梁の解析条件}
    \begin{enumerate}
      \item 一方を$x=0$,$y^2+z^2<10000[mm^2]$の範囲内に固定
      \item 長さ$L=200[mm]$,体積$V=50000[mm^3]$以下
      \item 材料はアルミ6061
      \item 断面は一様でなくとも,三次元的であってもよい
      \item 一般的な梁に見えなくてもよい
    \end{enumerate}
  \subsection{課題1}
    解析条件に沿って作成した片持ち梁の先端($x=200[mm]$)に,
    $Fz=-100[N]$の集中荷重を加えたとき,
    安全率4以上で,最もたわみにくい断面形状を設計せよ.
  \subsection{課題2}
    課題1と同様の片持ち梁の先端($x=200[mm]$)に,
    $Fy=100[N]$の集中荷重に変更したときの,
    最小安全率と最大たわみを示せ.
  \subsection{評価}
    課題1と課題2で使用した梁の設計がたわみにくいほど高く評価される.
    評価方法としては,課題1で上位30名を選出した後,
    課題2で決勝戦を行う.
