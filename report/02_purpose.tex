\section{方針}
  今回の課題は課題1を評価した後に課題2の評価に移る.
  つまり優勝を目指して課題2の評価を優先してしまうと,
  課題1の上位30名に選ばれなくなってしまう.
  そこで私は課題1を最優先として今回の課題に取り組む.\\\indent
  まず1.1の解析条件1より梁の固定される方の条件は以下のように表される.
  \begin{align}
    y^2+z^2&=10000[mm^2] \notag \\
    \iff \sqrt{y^2+z^2}&=100[mm] \label{eq:cond}
  \end{align}

  \eqref{eq:cond}より対角線が$100[mm]$未満の長方形の面積内に,
  固定される方が接地されなくてはならない.ここで対角線の長さを$d[mm]$とする.
  \begin{equation}
    d = \sqrt{y^2 + z^2} = 100[mm] \label{eq:diag}
  \end{equation}

  また課題1は梁の先端に集中荷重$Fz=-100[N]$の力をかける.
  このときの梁の中立軸は$y$軸と平行である.
  断面二次モーメントを$I$,梁の$y$軸方向の長さを$b$,
  $z$軸方向の長さを$h$とすると断面二次モーメントは以下のように表される.
  \begin{align}
    I=bh^3/12 \label{eq:MoI}
  \end{align}

  \eqref{eq:MoI}より$b$を大きくするより,
  $h$を大きくしたほうが断面二次モーメントが大きくなることがわかる.
  断面二次モーメントが大きいほど梁は曲がりにくい.
  つまり中立軸の外側に面積があればあるほど,梁は曲がりにくくなる.\\\indent
  以上より課題1を突破するためには,梁の設計は$z$軸方向に長く,
  梁の接地面の面積の対角線がなるべく$z$軸方向に長くなるようにすれば良い.
