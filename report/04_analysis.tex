\newpage
\section{考察}
  今回の実験は課題2を考慮せずに,課題1のみに焦点を当てて行った.
  課題1は$z$軸方向の集中荷重を梁に与えた際に,
  変位をなるべく小さくするという課題であった.\\\indent
  課題を達成するために解析条件から最も$z$軸方向に長い梁を作成した.
  3.1①のときは1点に集中して荷重がかかっていたため,
  変位が大きくなってしまった.
  そこで針の先端を丸めることにより,
  変位が小さくなった.
  これは集中荷重が1つの辺にかかっていたが,
  円周の面にかかったことによって,
  応力が分散され結果的に変位が小さくなった.\\\indent
  梁の先端を削った分の体積で,梁を補強したところ,
  変位が小さくなった.
  また補強したパーツを横長にするほど変位が小さくなったが,
  パーツの横の長さが$120[mm]$以上になると,変位が小さくならなくなった.
  つまり,応力が一番かかる接地面の角と,
  変位が一番大きくなる先端側を補強することによって,変位が小さくなったということである.
  そこで,一律に一つの直方体で補強するのではなく,
  接地面の角と梁の先端を補強するパーツを,
  それぞれ用意することで変位がもっと小さくなることが考えられる.\\\indent
  この梁で課題2の条件で実験を行うと安全率の最小値が,
  1.57と非常に低くなってしまった.
  しかし,課題1に焦点を当てて実験を行うという方針を達成できた.
